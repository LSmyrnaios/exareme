% Set the Page Layout
\documentclass[10pt]{article}
\usepackage[inner = 2.0cm, outer = 2.0cm, top = 2.0cm, bottom = 2.0cm]{geometry}

\usepackage{graphicx}
\usepackage{amsmath}
\usepackage{mathtools}
\usepackage{algorithm}
\usepackage[noend]{algpseudocode}

\DeclareMathOperator*{\log1}{log}

% Remove the Numbering of the Algorithm
\usepackage{caption}
\DeclareCaptionLabelFormat{algnonumber}{Algorithm}
\captionsetup[algorithm]{labelformat = algnonumber}

\begin{document}
\begin{algorithm}
	\caption{\underline{\textsc{ID3 Train}}}
	\label{id3}
	\begin{algorithmic}[1] % The number tells where the line numbering should start
		\Procedure{Global1}{}
				\State $\texttt{Tree} \gets \{\}$
		\EndProcedure
		\Loop
			\Procedure{Local1}{${X}^{(l)}, y^{(l)}$ } \Comment{run for $l = 1, \dots, L$}
					\State For each attribute \textbf{return} $\text{count}^{(l)}(x | y=C_m)$ for all $x$ values and all classes
			\EndProcedure
			\Procedure{Global2}{$\{ \text{count}^{(l)}(x| y=C_m) \} $}
					\State Sum local counts to obtain corresponding global counts
					\State Find data partition maximizing the Information Gain ($IG$) with probabilities computed from counts
					\State Add corresponding node to \texttt{Tree}
					\State \textbf{return} \texttt{Tree}
			\EndProcedure
			\Procedure{Local2}{${X}^{(l)}$, \texttt{Tree}}
					\State Split ${X}^{(l)}$ dataset according to \texttt{Tree}
			\EndProcedure
		\EndLoop
	\end{algorithmic}
\end{algorithm}

The information gain is defined as  $IG = H(y) - \sum_{\text{values of } x} p(x) H(y | x)$

where  $H(y | x) = - \sum_{\text{values of } y} p(y|x) \log1 p(y|x)$

The first term in IG is constant so we only need to compute the second for the minimization.


\end{document}
